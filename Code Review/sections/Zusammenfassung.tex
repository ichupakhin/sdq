\chapter{Zusammenfassung}
\label{ch:Zusammenfassung}



In einer modellgetriebenen Software-Entwicklung werden alle Artefakte des zu entwickelnden Systems als Modelle gesehen. Ein Performance-Modell ermöglicht Software-Entwicklern eine frühzeitige Analyse von programmierten Komponenten in Bezug auf Leistungseigenschaften, wie zum Beispiel Speicherbedarf oder Ausführungsdauer. Das Performance-Modell muss mit allen anderen im System vorhandenen Modellen konsistent gehalten werden. Falls ein Modell geändert wurde, muss das Performance-Modell aktualisiert werden. Allerdings ist eine Aktualisierung des Performance-Modells für große Systeme aufwändig. In einem agilen Software-Entwicklungsprozess wird das System mehrmals am Tag geändert, was eine Aktualisierung des Performance-Modelles nach jeder Änderung verursachen würde. Eine Lösung dafür wurde in \cite{mazkatli2018} vorgestellt und in \cite{mazkatli2020} weiterentwickelt. In dieser Bachelorarbeit implementieren wir einen Teil für diese Lösung. Wir extrahieren Code-Änderungen aus einem Commit und propagieren sie auf die bereits existierenden Quellcode-Modelle. Dabei achten wir darauf, dass die Quellcode-Modelle nur an den von den Änderungen betroffenen Stellen angepasst werden. Wir betrachten mehrere Ansätze, diskutieren ihre Vor- und Nachteile und implementieren einen für unsere Zwecke optimalen Ansatz. Wir evaluieren unsere Implementierung auf einem echten Open-Source-Projekt. 

%In der modellgetriebenen Software-Entwicklung werden alle Artefakte des zu entwickelnden Systems als Modelle gesehen. Ein Performance-Modell ermöglicht Software-Entwicklern eine frühzeitige Analyse von programmierten Komponenten in Bezug auf Leistungseigenschaften, wie zum Beispiel Speicherbedarf oder Ausführungsdauer. Das Einführen von Performance-Modellen in einen modernen Software-Entwicklungsprozess stellt eine Herausforderung dar. Eine der Schwierigkeiten ist eine Aktualisierung von Performance-Modellen. Eine Lösung dafür wurde in \cite{mazkatli2018} vorgestellt und in \cite{mazkatli2020} weiterentwickelt. In dieser Bachelorarbeit implementieren wir einen Teil für diese Lösung. Wir extrahieren Code-Änderungen aus einem Commit und propagieren diese auf die bereits existierenden Java-Code-Modelle. Dabei achten wir darauf, dass die Java-Code-Modelle nur an den von den Änderungen betroffenen Stellen angepasst werden. Wir betrachten mehrere Ansätze, diskutieren ihre Vor- und Nachteile und implementieren einen für unsere Zwecke optimalen Ansatz. Wir evaluieren unsere Implementierung auf einem echten Open-Source-Projekt.
